% Tips with rmarkdown
% author : Sébastien Rochette
% contact : sebastienrochettefr@gmail.com
% website : https://statnmap.com

% -- Command to find which language is loaded in babel -- %
% http://tex.stackexchange.com/questions/287667/ifpackagewith-doesnt-behave-as-i-expected-with-global-options
\usepackage{xparse}
\ExplSyntaxOn
\NewDocumentCommand{\packageoptionsTF}{mmmm}
 {
  \stanton_package_options:nnTF { #1 } { #2 } { #3 } { #4 }
 }

\cs_new_protected:Nn \stanton_package_options:nnTF
 {
  \clist_map_inline:nn { #2 }
   {
    \clist_if_in:cnTF { opt@#1.sty } { ##1 }
     { #3 } % it's a local option
     {
      \clist_if_in:cnTF { @classoptionslist } { ##1 }
       { #3 } % it's a global option
       { #4 }
     }
   }
}
\ExplSyntaxOff

% -- Define a variable depending on language -- %
\newcommand{\Lang}{0}

\makeatletter
\@ifpackageloaded{babel}{
  \packageoptionsTF{babel}{english}{%
    \renewcommand{\Lang}{1}% english
  }{%
    \renewcommand{\Lang}{2}% french
  }
}{}
\makeatother

% -- Define specific lateX options depending on language -- %
\ifnum\Lang = 1
  \usepackage{enumitem}
  \setlist{itemsep = 0pt}
  \setlist{topsep = 0pt}
\fi
\ifnum\Lang = 2
%
\fi

% -- Texte coloration setups -- %
\usepackage{xcolor}
\definecolor{blueSeb}{RGB}{30,115,190}
\definecolor{advertcolor}{HTML}{FF8929}
% \definecolor{blueShaded}{HTML}{D6E8F5}

\definecolor{backcolor}{RGB}{235, 235, 235}
\newcommand{\mybox}[1]{\par\noindent\colorbox{backcolor}
  {\parbox{\dimexpr\textwidth-2\fboxsep\relax}{#1}}}

\newcommand{\advert}[1]{\textit{\textcolor{advertcolor}{#1}}}
\newcommand{\codecommand}[1]{\texttt{\colorbox{backcolor}{#1}}}

% - source: http://tex.stackexchange.com/questions/82028/how-do-i-create-a-variant-of-the-snugshade-box-from-the-framed-package-to-wrap-m
\newenvironment{blueShaded}[1][D6E8F5]{
  \definecolor{shadecolor}{HTML}{#1}%
  \begin{snugshade}%
}{%
    \end{snugshade}%
}

% -- Sections format -- %
%% Provide a definition to \subparagraph to keep titlesec happy
\let\subparagraph\oldsubparagraph
\let\paragraph\oldparagraph
%% Load titlesec
%\usepackage[compact]{titlesec}
\usepackage{titlesec}
%% Revert \subparagraph to the Rmd definition
% Redefines (sub)paragraphs to behave more like sections
\ifx\paragraph\undefined\else
\let\oldparagraph\paragraph
\renewcommand{\paragraph}[1]{\oldparagraph{#1}\mbox{}}
\fi
\ifx\subparagraph\undefined\else
\let\oldsubparagraph\subparagraph
\renewcommand{\subparagraph}[1]{\oldsubparagraph{#1}\mbox{}}
\fi


%\titleformat{(command)}[(shape)]{(format)}{(label)}{(sep)}{(before)}[(after)]
\titleformat{\section}%
[hang]% style du titre (hang, display, runin, leftmargin, drop, wrap)
{\Large\color[RGB]{30,115,190}}%changement de fonte commun au numéro et au titre
{\thesection.}% spécification du numéro
{1em}% 1em espace entre le numéro et le titre
{}% changement de fonte du titre
  
\titleformat{\subsection}%
[hang]% style du titre (hang, display, runin, leftmargin, drop, wrap)
{\large\bfseries} %\itshape {\Large\bfseries\color[RGB]{30,115,190}}%changement de fonte commun au numéro et au titre
{\thesubsection.}% spécification du numéro
{1em}% {1em}% espace entre le numéro et le titre
{}% changement de fonte du titre

%\titlespacing*{(command)}{(left)}{(beforesep)}{(aftersep)}[(right)]
\titlespacing*{\section}{0em}{*4}{*1}



% -- Other lateX setup -- %
\hypersetup{pdfauthor=StatnMap, pdftitle=Tips with Rmarkdown, pdfkeywords=R-rmarkdown, pdfcreator=pdflatex}
\hypersetup{linktocpage=true, colorlinks=true, linkcolor=[RGB]{30,115,190}, citecolor=[RGB]{30,115,190}, filecolor=[RGB]{30,115,190}, urlcolor=[RGB]{30,115,190}}

\setcounter{section}{0} % Value for first section

